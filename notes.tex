\documentclass{amsart}

\newtheorem*{theorem}{Theorem}
\newtheorem*{lemma}{Lemma}
\newtheorem*{axiom}{Axiom}

\theoremstyle{definition}
\newtheorem*{definition}{Definition}

\newcommand{\R}{\mathbb{R}}
\newcommand{\abs}[1]{|#1|}

\title{MATH 355: Notes}
\author{Alexander Lee}

\begin{document}

\maketitle

\section*{The Real Numbers}

\subsection*{Some Preliminaries}

\begin{theorem}
  Two real numbers $a$ and $b$ are equal if and only if for every real number
  $\epsilon > 0$ it follows that $\abs{a-b} < \epsilon$.
\end{theorem}

\subsection*{The Axiom of Completeness}

\begin{axiom}[Axiom of Completeness]
  Every nonempty set of real numbers that is bounded above has a least upper
  bound.
\end{axiom}

\begin{definition}
  A set $A \subseteq R$ is \emph{bounded above} if there exists a number $b \in
  \R$ such that $a \le b$ for all $a \in A$. The number $b$ is called an
  \emph{upper bound} for $A$.

  Similarly, the set $A$ is \emph{bounded below} if there exists a \emph{lower
  bound} $l \in R$ satisfying $l \le a$ for every $a \in A$.
\end{definition}

\begin{definition}
  A real number $s$ is the \emph{least upper bound} for a set $A \subseteq \R$
  if it meets the following two criteria:
  \begin{enumerate}
    \item $s$ is an upper bound for $A$;
    \item if $b$ is any upper bound for $A$, then $s \le b$.
  \end{enumerate}
  The least upper bound is also frequently called the \emph{supremum} of the set
  $A$. We write $s = \sup{A}$ for the least upper bound.

  The \emph{greatest lower bound} or \emph{infimum} for $A$ is defined in a
  similar way and is denoted by $\inf{A}$.
\end{definition}

\begin{definition}
  A real number $a_0$ is a \emph{maximum} of the set $A$ if $a_0$ is an element
  of $A$ and $a_0 \ge a$ for all $a \in A$. Similarly, a number $a_1$ is a
  \emph{minimum} of $A$ if $a_1 \in A$ and $a_1 \le a$ for every $a \in A$.
\end{definition}

\begin{lemma}
  Assume $s \in \R$ is an upper bound for a set $A \subseteq R$. Then, $s =
  \sup{A}$ if and only if, for every choice of $\epsilon > 0$, there exists an
  element $a \in A$ satisfying $s - \epsilon < a$.
\end{lemma}

\end{document}
