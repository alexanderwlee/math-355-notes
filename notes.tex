\documentclass{amsart}

\usepackage{enumitem}

\newtheorem*{theorem}{Theorem}
\newtheorem*{lemma}{Lemma}
\newtheorem*{axiom}{Axiom}

\theoremstyle{definition}
\newtheorem*{definition}{Definition}

\newcommand{\N}{\mathbb{N}}
\newcommand{\Q}{\mathbb{Q}}
\newcommand{\R}{\mathbb{R}}
\newcommand{\abs}[1]{|#1|}
\newcommand{\st}{\mathrel{:}}

\title{MATH 355: Notes}
\author{Alexander Lee}

\begin{document}

\maketitle

\section*{1 The Real Numbers}

\subsection*{1.2 Some Preliminaries}

\begin{theorem}[Triangle Inequality]
  For all choices of $a$ and $b$, $\abs{a + b} \le \abs{a} + \abs{b}$.
\end{theorem}

\begin{theorem}
  Two real numbers $a$ and $b$ are equal if and only if for every real number
  $\epsilon > 0$ it follows that $\abs{a-b} < \epsilon$.
\end{theorem}

\subsection*{1.3 The Axiom of Completeness}

\begin{axiom}[Axiom of Completeness]
  Every nonempty set of real numbers that is bounded above has a least upper
  bound.
\end{axiom}

\begin{definition}
  A set $A \subseteq \R$ is \emph{bounded above} if there exists a number $b \in
  \R$ such that $a \le b$ for all $a \in A$. The number $b$ is called an
  \emph{upper bound} for $A$.

  Similarly, the set $A$ is \emph{bounded below} if there exists a \emph{lower
  bound} $l \in \R$ satisfying $l \le a$ for every $a \in A$.
\end{definition}

\begin{definition}
  A real number $s$ is the \emph{least upper bound} for a set $A \subseteq \R$
  if it meets the following two criteria:
  \begin{enumerate}[label={(\roman*)}]
    \item $s$ is an upper bound for $A$;
    \item if $b$ is any upper bound for $A$, then $s \le b$.
  \end{enumerate}
  The least upper bound is also frequently called the \emph{supremum} of the set
  $A$. We write $s = \sup(A)$ for the least upper bound.

  The \emph{greatest lower bound} or \emph{infimum} for $A$ is defined in a
  similar way and is denoted by $\inf{A}$.
\end{definition}

\begin{theorem}
  Let $A \subseteq \R$ be bounded above and below. Then, the $\sup(A)$ and
  $\inf{A}$ are unique.
\end{theorem}

\begin{definition}
  A real number $a_0$ is a \emph{maximum} of the set $A$ if $a_0$ is an element
  of $A$ and $a_0 \ge a$ for all $a \in A$. Similarly, a number $a_1$ is a
  \emph{minimum} of $A$ if $a_1 \in A$ and $a_1 \le a$ for every $a \in A$.
\end{definition}

\begin{theorem}
  Let $A \subseteq \R$ be nonempty and bounded above.
  \begin{enumerate}[label={(\roman*)}]
    \item Let $c \in \R$ and define the set $c + A$ by
      \[
        c + A = \{c + a \st a \in A\}.
      \]
      Then $\sup(c + A) = c + \sup(A)$.
    \item Let $c \in \R$ with $c > 0$ and define the set $cA$ by
      \[
        cA = \{ca \st a \in A\}.
      \]
      Then $\sup(cA) = c \sup(A)$.
  \end{enumerate}
\end{theorem}

\begin{lemma}
  Assume $s \in \R$ is an upper bound for a set $A \subseteq R$. Then, $s =
  \sup(A)$ if and only if, for every choice of $\epsilon > 0$, there exists an
  element $a \in A$ satisfying $s - \epsilon < a$.
\end{lemma}

\subsection*{1.4 Consequences of Completeness}

\begin{theorem}[Nested Interval Property]
  For each $n \in \N$, assume we are given a closed interval $I_n = [a_n, b_n] =
  \{x \in \R \st a_n \le x \le b_n\}$. Assume also that each $I_n$ contains
  $I_{n+1}$. Then, the resulting nested sequence of closed intervals
  \[
    I_1 \supseteq I_2 \supseteq I_3 \supseteq I_4 \supseteq \cdots
  \]
  has a nonempty intersection; that is, $\bigcap_{n=1}^{\infty} I_n \neq
  \emptyset$.
\end{theorem}

\begin{theorem}[Archimedean Property]
  \begin{enumerate}[label={(\roman*)}]
    \item Given any number $x \in \R$, there exists an $n \in \N$ satisfying $n
      > x$.
    \item Given any real number $y > 0$, there exists an $n \in \N$ satisfying
      $1/n < y$.
  \end{enumerate}
\end{theorem}

\begin{definition}
  A set $X$ is \emph{dense in $\R$} if for any $a, b \in \R$ with $a < b$,
  $\exists x \in X$ with $a < x < b$.
\end{definition}

\begin{theorem}[Density of $\Q$ in $\R$]
  For every two real numbers $a$ and $b$ with $a < b$, there exists a rational
  number $r$ satisfying $a < r < b$.
\end{theorem}

\subsection*{1.5 Cardinality}

\begin{definition}
  A function $f : A \rightarrow B$ is \emph{1--1 (injective)} if for all $a_1,
  a_2 \in A$, $f(a_1) = f(a_2)$ implies that $a_1 = a_2$.
\end{definition}

\begin{definition}
  A function $f : A \rightarrow B$ is \emph{onto (surjective)} if for all $b
  \in B$, there exists an $a \in A$ such that $f(a) = b$.
\end{definition}

\begin{definition}
  A function $f : A \rightarrow B$ is a \emph{bijection} if it is both 1--1 and
  onto.
\end{definition}

\begin{definition}
  Two sets $A$ and $B$ have the same \emph{cardinality} if there exists a
  bijection $f : A \rightarrow B$. In this case, we write $A \sim B$.
\end{definition}

\begin{definition}
  A set $A$ is \emph{finite} if there exists an $n \in \N$ such that $A \sim
  \{1, 2, \ldots, n\}$.
\end{definition}

\begin{definition}
  A set $A$ is \emph{countable} if $A \sim \N$.
\end{definition}

\begin{definition}
  A set which is not finite nor countable is \emph{uncountable}.
\end{definition}

\begin{theorem}
  \begin{enumerate}[label={(\roman*)}]
    \item The set $\Q$ is countable.
    \item The set $\R$ is uncountable.
  \end{enumerate}
\end{theorem}

\end{document}
