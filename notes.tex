\documentclass{amsart}

\usepackage{enumitem}
\usepackage{hyperref}

\newtheorem*{theorem}{Theorem}
\newtheorem*{lemma}{Lemma}
\newtheorem*{corollary}{Corollary}
\newtheorem*{axiom}{Axiom}

\theoremstyle{definition}
\newtheorem*{definition}{Definition}

\newcommand{\N}{\mathbb{N}}
\newcommand{\Q}{\mathbb{Q}}
\newcommand{\R}{\mathbb{R}}
\newcommand{\abs}[1]{|#1|}
\newcommand{\st}{\mathrel{:}}

\title{MATH 355: Notes}
\author{Alexander Lee}

\begin{document}

\maketitle

\section*{1 The Real Numbers}

\subsection*{1.2 Some Preliminaries}

\begin{theorem}[Triangle Inequality]
  For all choices of $a$ and $b$, $\abs{a + b} \le \abs{a} + \abs{b}$.
\end{theorem}

\begin{theorem}
  Two real numbers $a$ and $b$ are equal if and only if for every real number
  $\epsilon > 0$ it follows that $\abs{a-b} < \epsilon$.
\end{theorem}

\subsection*{1.3 The Axiom of Completeness}

\begin{axiom}[Axiom of Completeness]
  Every nonempty set of real numbers that is bounded above has a least upper
  bound.
\end{axiom}

\begin{definition}
  A set $A \subseteq \R$ is \emph{bounded above} if there exists a number $b \in
  \R$ such that $a \le b$ for all $a \in A$. The number $b$ is called an
  \emph{upper bound} for $A$.

  Similarly, the set $A$ is \emph{bounded below} if there exists a \emph{lower
  bound} $l \in \R$ satisfying $l \le a$ for every $a \in A$.
\end{definition}

\begin{definition}
  A real number $s$ is the \emph{least upper bound} for a set $A \subseteq \R$
  if it meets the following two criteria:
  \begin{enumerate}[label={(\roman*)}]
    \item $s$ is an upper bound for $A$;
    \item if $b$ is any upper bound for $A$, then $s \le b$.
  \end{enumerate}
  The least upper bound is also frequently called the \emph{supremum} of the set
  $A$. We write $s = \sup(A)$ for the least upper bound.

  The \emph{greatest lower bound} or \emph{infimum} for $A$ is defined in a
  similar way and is denoted by $\inf(A)$.
\end{definition}

\begin{theorem}
  Let $A \subseteq \R$ be bounded above and below. Then, the $\sup(A)$ and
  $\inf(A)$ are unique.
\end{theorem}

\begin{definition}
  A real number $a_0$ is a \emph{maximum} of the set $A$ if $a_0$ is an element
  of $A$ and $a_0 \ge a$ for all $a \in A$. Similarly, a number $a_1$ is a
  \emph{minimum} of $A$ if $a_1 \in A$ and $a_1 \le a$ for every $a \in A$.
\end{definition}

\begin{theorem}
  Let $A \subseteq \R$ be nonempty and bounded above.
  \begin{enumerate}[label={(\roman*)}]
    \item Let $c \in \R$ and define the set $c + A$ by
      \[
        c + A = \{c + a \st a \in A\}.
      \]
      Then $\sup(c + A) = c + \sup(A)$.
    \item Let $c \in \R$ with $c > 0$ and define the set $cA$ by
      \[
        cA = \{ca \st a \in A\}.
      \]
      Then $\sup(cA) = c \sup(A)$.
  \end{enumerate}
\end{theorem}

\begin{lemma}
  Assume $s \in \R$ is an upper bound for a set $A \subseteq R$. Then, $s =
  \sup(A)$ if and only if, for every choice of $\epsilon > 0$, there exists an
  element $a \in A$ satisfying $s - \epsilon < a$.
\end{lemma}

\subsection*{1.4 Consequences of Completeness}

\begin{theorem}[Nested Interval Property]
  For each $n \in \N$, assume we are given a closed interval $I_n = [a_n, b_n] =
  \{x \in \R \st a_n \le x \le b_n\}$. Assume also that each $I_n$ contains
  $I_{n+1}$. Then, the resulting nested sequence of closed intervals
  \[
    I_1 \supseteq I_2 \supseteq I_3 \supseteq I_4 \supseteq \cdots
  \]
  has a nonempty intersection; that is, $\bigcap_{n=1}^{\infty} I_n \neq
  \emptyset$.
\end{theorem}

\begin{theorem}[Archimedean Property]
  \begin{enumerate}[label={(\roman*)}]
    \item Given any number $x \in \R$, there exists an $n \in \N$ satisfying $n
      > x$.
    \item Given any real number $y > 0$, there exists an $n \in \N$ satisfying
      $1/n < y$.
  \end{enumerate}
\end{theorem}

\begin{definition}
  A set $X$ is \emph{dense in $\R$} if for any $a, b \in \R$ with $a < b$,
  $\exists x \in X$ with $a < x < b$.
\end{definition}

\begin{theorem}[Density of $\Q$ in $\R$]
  For every two real numbers $a$ and $b$ with $a < b$, there exists a rational
  number $r$ satisfying $a < r < b$.
\end{theorem}

\subsection*{1.5 Cardinality}

\begin{definition}
  A function $f : A \rightarrow B$ is \emph{1--1 (injective)} if for all $a_1,
  a_2 \in A$, $f(a_1) = f(a_2)$ implies that $a_1 = a_2$.
\end{definition}

\begin{definition}
  A function $f : A \rightarrow B$ is \emph{onto (surjective)} if for all $b
  \in B$, there exists an $a \in A$ such that $f(a) = b$.
\end{definition}

\begin{definition}
  A function $f : A \rightarrow B$ is a \emph{bijection} if it is both 1--1 and
  onto.
\end{definition}

\begin{definition}
  Two sets $A$ and $B$ have the same \emph{cardinality} if there exists a
  bijection $f : A \rightarrow B$. In this case, we write $A \sim B$.
\end{definition}

\begin{definition}
  A set $A$ is \emph{finite} if there exists an $n \in \N$ such that $A \sim
  \{1, 2, \ldots, n\}$.
\end{definition}

\begin{definition}
  A set $A$ is \emph{countable} if $A \sim \N$.
\end{definition}

\begin{definition}
  A set which is not finite nor countable is \emph{uncountable}.
\end{definition}

\begin{theorem}
  \begin{enumerate}[label={(\roman*)}]
    \item The set $\Q$ is countable.
    \item The set $\R$ is uncountable.
  \end{enumerate}
\end{theorem}

\begin{theorem}
  If $A \subseteq B$ is countable, then $A$ is either countable or finite.
\end{theorem}

\begin{theorem}
  \begin{enumerate}[label={(\roman*)}]
    \item If $A_1, A_2, \ldots, A_m$ are each countable sets, then the union
      $A_1 \cup A_2 \cup \cdots \cup A_m$ is countable.
    \item If $A_n$ is a countable set for each $n \in \N$, then
      $\bigcup_{n=1}^{\infty} A_n$ is countable.
  \end{enumerate}
\end{theorem}

\subsection*{1.6 Cantor's Theorem}

\begin{theorem}
  The open interval $(0, 1) = \{x \in \R \st 0 < x < 1\}$ is uncountable.
\end{theorem}

\begin{definition}
  Given a set $A$, the \emph{power set} $P(A)$ refers to the collection of all
  subsets of $A$.
\end{definition}

\begin{theorem}[Cantor's Theorem]
  Given any set $A$, there does not exist a function $f : A \rightarrow P(A)$
  that is onto.
\end{theorem}

\section*{2 Sequences and Series}

\subsection*{2.2 The Limit of a Sequence}

\begin{definition}
  A \emph{sequence} is a function whose domain is $\N$.
\end{definition}

\begin{definition}[Convergence of a Sequence]
  A sequence $(a_n)$ \emph{converges} to a real number $a$ if, for every
  positive number $\epsilon$, there exists and $N \in \N$ such that whenever $n
  \ge N$, it follows that $\abs{a_n - a} < \epsilon$.
\end{definition}

\begin{definition}
  Given a real number $a \in \R$ and a positive number $\epsilon > 0$, the set
  \[
    V_\epsilon(a) = \{x \in \R \st \abs{x - a} < \epsilon\} = (a - \epsilon, a +
    \epsilon)
  \]
  is called the \emph{$\epsilon$-neighborhood of $a$}.
\end{definition}

\begin{definition}[Convergence of a Sequence: Topological Version]
  A sequence $(a_n)$ converges to $a$ if, given any $\epsilon$-neighborhood
  $V_\epsilon(a)$ of $a$, there exists a point in the sequence after which all
  of the terms are in $V_\epsilon(a)$. In other words, every
  $\epsilon$-neighborhood contains all but a finite number of terms of $(a_n)$.
\end{definition}

\begin{theorem}[Uniqueness of Limits]
  The limit of a sequence, when it exists, must be unique.
\end{theorem}

\begin{definition}
  A sequence that does not converge is said to \emph{diverge}.
\end{definition}

\subsection*{2.3 The Algebraic and Order Limit Theorems}

\begin{definition}
  A sequence $(x_n)$ is \emph{bounded} if there exists a number $M > 0$ such
  that $\abs{x_n} \le M$ for all $n \in \N$.
\end{definition}

\begin{theorem}
  Every convergent sequence is bounded.
\end{theorem}

\begin{theorem}[Algebraic Limit Theorem]
  Let $\lim{a_n} = a$ and $\lim{b_n} = b$. Then,
  \begin{enumerate}[label={(\roman*)}]
    \item $\lim(ca_n) = ca$, for all $c \in \R$;
    \item $\lim(a_n + b_n) = a + b$;
    \item $\lim(a_n b_n) = ab$;
    \item $\lim(a_n / b_n) = a / b$, provided $b \neq 0$.
  \end{enumerate}
\end{theorem}

\begin{theorem}[Order Limit Theorem]
  Assume $\lim{a_n} = a$ and $\lim{b_n} = b$.
  \begin{enumerate}[label={(\roman*)}]
    \item If $a_n \ge 0$ for all $n \in \N$, then $a \ge 0$.
    \item If $a_n \le b_n$ for all $n \in \N$, then $a \le b$.
    \item If there exists $c \in \R$ for which $c \le b_n$ for all $n \in \N$,
      then $c \le b$. Similarly, if $a_n \le c$ for all $n \in \N$, then $a \le
      c$.
  \end{enumerate}
\end{theorem}

\subsection*{2.4 The Monotone Convergence Theorem and a First Look at Infinite
Series}

\begin{definition}
  A sequence $(a_n)$ is \emph{increasing} if $a_n \le a_{n+1}$ for all $n \in
  \N$ and \emph{decreasing} if $a_n \ge a_{n+1}$ for all $n \in \N$. A sequence
  is \emph{monotone} if it is either increasing or decreasing.
\end{definition}

\begin{theorem}[Monotone Convergence Theorem]
  If a sequence is monotone and bounded, then it converges.
\end{theorem}

\begin{definition}[Convergence of a Series]
  Let $(b_n)$ be a sequence. An \emph{infinite series} is a formal expression of
  the form
  \[
    \sum_{n=1}^{\infty} b_n = b_1 + b_2 + b_3 + b_4 + b_5 + \cdots.
  \]
  We definite the corresponding \emph{sequence of partial sums} $(s_m)$ by
  \[
    s_m = b_1 + b_2 + b_3 + \cdots + b_m,
  \]
  and say that the series $\sum_{n=1}^{\infty} b_n$ \emph{converges to $B$} if
  the sequence $(s_m)$ converges to $B$. In this case, we write
  $\sum_{n=1}^{\infty} b_n = B$.
\end{definition}

\begin{theorem}[Cauchy Condensation Test]
  Suppose $(b_n)$ is decreasing and satisfies $b_n \ge 0$ for all $n \in \N$.
  Then, the series $\sum_{n=1}^{\infty} b_n$ converges if and only if the series
  \[
    \sum_{n=0}^{\infty} 2^n b_{2^n} = b_1 + 2 b_2 + 4 b_4 + 8 b_8 + 16 b_{16} +
    \cdots
  \]
  converges.
\end{theorem}

\begin{corollary}
  The series $\sum_{n=1}^\infty 1 / n^p$ converges if and only if $p > 1$.
\end{corollary}

\subsection*{2.5 Subsequences and the Bolzano-Weierstrass Theorem}

\begin{definition}
  Let $(a_n)$ be a sequence of real numbers, and let $n_1 < n_2 < n_3 < n_4 <
  n_5 < \ldots$ be an increasing sequence of natural numbers. Then the sequence
  \[
    (a_{n_1}, a_{n_2}, a_{n_3}, a_{n_4}, a_{n_5}, \ldots)
  \]
  is called a \emph{subsequence} of $(a_n)$ and is denoted by $(a_{n_k})$, where
  $k \in \N$ indexes the subsequence.
\end{definition}

\begin{theorem}
  Subsequences of a convergent sequence converge to the same limit as the
  original sequence.
\end{theorem}

\begin{corollary}[Divergence Criterion]
  Suppose that $(a_n)$ is a sequence and $(a_{n_k})$ is a subsequence that
  diverges, then $(a_n)$ diverges. If $(a^1_{n_k})$ and $(a^2_{n_k})$ converge
  to $a^1$ and $a^2$ with $a^1 \neq a^2$, then $(a_n)$ diverges.
\end{corollary}

\begin{theorem}[Bolzano-Weierstrass Theorem]
  Every bounded sequence contains a convergent subsequence.
\end{theorem}

\subsection*{2.6 The Cauchy Criterion}

\begin{definition}
  A sequence $(a_n)$ is called a \emph{Cauchy sequence} if, for every $\epsilon
  > 0$, there exists an $N \in \N$ such that whenever $m, n \ge N$ it follows
  that $\abs{a_n - a_m} < \epsilon$.
\end{definition}

\begin{theorem}
  Every convergence sequence is a Cauchy sequence.
\end{theorem}

\begin{lemma}
  Cauchy sequences are bounded.
\end{lemma}

\begin{theorem}[Cauchy Criterion]
  A sequence converges if and only if it is a Cauchy sequence.
\end{theorem}

\subsection*{2.7 Properties of Infinite Series}

\begin{theorem}[Algebraic Limit Theorem for Series]
  If $\sum_{k=1}^\infty a_k = A$ and $\sum_{k=1}^\infty b_k = B$, then
  \begin{enumerate}[label={(\roman*)}]
    \item $\sum_{k=1}^\infty c a_k = c A$ for all $c \in \R$ and
    \item $\sum_{k=1}^\infty (a_k + b_k) = A + B$.
  \end{enumerate}
\end{theorem}

\begin{theorem}[Cauchy Criterion for Series]
  The series $\sum_{k=1}^\infty a_k$ converges if and only if, given $\epsilon >
  0$, there exists an $N \in \N$ such that whenever $n > m \ge N$ it follows
  that
  \[
    \abs{a_{m+1} + a_{m+2} + \cdots + a_{n}} < \epsilon.
  \]
\end{theorem}

\begin{theorem}[Divergence Test]
  If the series $\sum_{k=1}^\infty a_k$ converges, then $(a_k) \rightarrow 0$.
  Equivalently, if $(a_k) \not\rightarrow 0$, then $\sum_{k=1}^\infty a_k$
  diverges.
\end{theorem}

\begin{theorem}[Comparison Test]
  Assume $(a_k)$ and $(b_k)$ are sequences satisfying $0 \le a_k \le b_k$ for
  all $k \in \N$.
  \begin{enumerate}[label={(\roman*)}]
    \item If $\sum_{k=1}^\infty b_k$ converges, then $\sum_{k=1}^\infty a_k$
      converges.
    \item If $\sum_{k=1}^\infty a_k$ diverges, then $\sum_{k=1}^\infty b_k$
      diverges.
  \end{enumerate}
\end{theorem}

\begin{theorem}[Squeeze Theorem]
  Suppose $a_n \le b_n \le c_n$ for all $n \in \N$, and if $\lim a_n = \lim c_n
  = l$, then $\lim b_n = l$ as well.
\end{theorem}

\begin{definition}[Geometric Series]
  A series is called \emph{geometric} if it is of the form
  \[
    \sum_{k=0}^\infty ar^k = a + ar + ar^2 + ar^3 + \cdots.
  \]
\end{definition}

\begin{theorem}
  $\sum_{k=0}^\infty ar^k = \frac{a}{1 - r}$ if and only if $\abs{r} < 1$.
\end{theorem}

\begin{theorem}[Absolute Convergence Test]
  If the series $\sum_{n=1}^\infty \abs{a_n}$ converges, then $\sum_{n=1}^\infty
  a_n$ converges as well.
\end{theorem}

\begin{theorem}[Alternating Series Test]
  Let $(a_n)$ be a sequence satisfying,
  \begin{enumerate}[label={(\roman*)}]
    \item $a_1 \ge a_2 \ge a_3 \ge \cdots \ge a_n \ge a_{n+1} \ge \cdots$ and
    \item $(a_n) \rightarrow 0$.
  \end{enumerate}
  Then, the alternating series $\sum_{n=1}^\infty {(-1)}^{n+1} a_n$ converges.
\end{theorem}

\begin{definition}
  If $\sum_{n=1}^\infty \abs{a_n}$ converges, then we say that the original
  series $\sum_{n=1}^\infty a_n$ \emph{converges absolutely}. If, on the other
  hand, the series $\sum_{n=1}^\infty a_n$ converges but the series of absolute
  values $\sum_{n=1}^\infty \abs{a_n}$ does not converge, then we say that the
  original series $\sum_{n=1}^\infty a_n$ \emph{converges conditionally}.
\end{definition}

\section*{3 Basic Topology of $\R$}

\subsection*{3.2 Open and Closed Sets}

\begin{definition}
  A set $O \subseteq \R$ is \emph{open} if for all points $a \in O$ there exists
  an $\epsilon$-neighborhood $V_\epsilon(a) \subseteq O$.
\end{definition}

\begin{theorem}
  \begin{enumerate}[label={(\roman*)}]
    \item The union of an arbitrary collection of open sets is open.
    \item The intersection of a finite collection of open sets is open.
  \end{enumerate}
\end{theorem}

\begin{definition}
  A point $x$ is a \emph{limit point} of a set $A$ if every
  $\epsilon$-neighborhood $V_\epsilon(x)$ of $x$ intersects the set $A$ at some
  point other than $x$.
\end{definition}

\begin{theorem}
  A point $x$ is a limit point of a set $A$ if and only if $x = \lim a_n$ for
  some sequence $(a_n)$ contained in $A$ satisfying $a_n \neq x$ for all $n \in
  \N$.
\end{theorem}

\begin{definition}
  A point $a \in A$ is an \emph{isolated point of $A$} if it is not a limit
  point of $A$.
\end{definition}

\begin{definition}
  A set $F \subseteq \R$ is \emph{closed} if it contains its limit points.
\end{definition}

\begin{theorem}
  A set $F \subset \R$ is closed if and only if every Cauchy sequence contained
  in $F$ has a limit that is also an element of $F$.
\end{theorem}

\begin{definition}
  Given a set $A \subseteq \R$, let $L$ be the set of all limit points of $A$.
  The \emph{closure} of $A$ is defined to be $\overline{A} = A \cup L$.
\end{definition}

\begin{theorem}
  For any $A \subseteq \R$, the closure $\overline{A}$ is a closed set and is the
  smallest closed set containing $A$.
\end{theorem}

\begin{theorem}
  A set $O$ is open if and only if $O^c$ is closed. Likewise, a set $F$ is
  closed if and only if $F^c$ is open.
\end{theorem}

\begin{theorem}
  \begin{enumerate}[label={(\roman*)}]
    \item The union of a finite collection of closed sets is closed.
    \item The intersection of an arbitrary collection of closed sets is closed.
  \end{enumerate}
\end{theorem}

\subsection*{3.3 Compact Sets}

\begin{definition}[Compactness]
  A set $K \subseteq \R$ is \emph{compact} if every sequence in $K$ has a
  subsequence that converges to a limit that is also in $K$.
\end{definition}

\begin{definition}
  A set $A \subseteq \R$ is \emph{bounded} if there exists $M > 0$ such that
  $\abs{a} \le M$ for all $a \in A$.
\end{definition}

\begin{theorem}[Characterization of Compactness in $\R$]
  A set $K \subseteq \R$ is compact if and only if it is closed and bounded.
\end{theorem}

\begin{theorem}[Nested Compact Set Property]
  If
  \[
    K_1 \supseteq K_2 \supseteq K_3 \supseteq K_4 \supseteq \cdots
  \]
  is a nested sequence of nonempty compact sets, then the intersection
  $\cap_{n=1}^\infty K_n$ is not empty.
\end{theorem}

\begin{definition}
  Let $A \subseteq \R$. An \emph{open cover} for $A$ is a (possibly infinite)
  collection of open sets $\{O_\lambda \st \lambda \in \Lambda\}$ whose union
  contains the set $A$; that is, $A \subseteq \cup_{\lambda \in \Lambda}
  O_\lambda$. Given an open cover for $A$, a \emph{finite subcover} is a finite
  subcollection of open sets from the original open cover whose union still
  manages to completely contain $A$.
\end{definition}

\begin{theorem}[Heine-Borel Theorem]
  Let $K$ be a subset of $\R$. All of the following statements are equivalent in
  the sense that any one of them implies the two others:
  \begin{enumerate}[label={(\roman*)}]
    \item $K$ is compact.
    \item $K$ is closed and bounded.
    \item Every open cover for $K$ has a finite subcover.
  \end{enumerate}
\end{theorem}

\section*{4 Functional Limits and Continuity}

\subsection*{4.2 Functional Limits}

\begin{definition}[Functional Limit]
  Let $f : A \to \R$, and let $c$ be a limit point of the domain $A$. We say
  that $\lim_{x \to c} f(x) = L$ provided that, for all $\epsilon > 0$, there
  exists a $\delta > 0$ such that whenever $0 < \abs{x - c} < \delta$ (and $x
  \in A$) it follows that $\abs{f(x) - L} < \epsilon$.
\end{definition}

\begin{definition}[Functional Limit: Topological Version]
  Let $c$ be a limit point of the domain of $f : A \to \R$. We say $\lim_{x \to
  c} f(x) = L$ provided that, for every $\epsilon$-neighborhood $V_\epsilon(L)$
  of $L$, there exists a $\delta$-neighborhood $V_\delta(c)$ around $c$ with
  the property that for all $x \in V_\delta(c)$ different from $c$ (with $x \in
  A$) it follows that $f(x) \in V_\epsilon(L)$.
\end{definition}

\begin{theorem}[Sequential Criterion for Functional Limits]
  Given a function $f : A \to \R$ and a limit point $c$ of $A$, the following
  two statements are equivalent:
  \begin{enumerate}[label={(\roman*)}]
    \item $\lim_{x \to c} f(x) = L$.
    \item For all sequences $(x_n) \subseteq A$ satisfying $x_n \neq c$ and
      $(x_n) \to c$, it follows that $f(x_n) \to L$.
  \end{enumerate}
\end{theorem}

\begin{corollary}[Algebraic Limit Theorem for Functional Limits]
  Let $f$ and $g$ be functions defined on a domain $A \subseteq \R$, and assume
  $\lim_{x \to c} f(x) = L$ and $\lim_{x \to c} g(x) = M$ for some limit point
  $c$ of $A$. Then,
  \begin{enumerate}[label={(\roman*)}]
    \item $\lim_{x \to c} k f(x) = k L$ for all $k \in \R$,
    \item $\lim_{x \to c} [f(x) + g(x)] = L + M$,
    \item $\lim_{x \to c} [f(x) g(x)] = L M$, and
    \item $\lim_{x \to c} f(x) / g(x) = L / M$, provided $M \neq 0$.
  \end{enumerate}
\end{corollary}

\begin{corollary}[Divergence Criterion for Functional Limits]
  Let $f$ be a function defined on $A$, and let $c$ be a limit point of $A$. If
  there exists two sequences $(x_n)$ and $(y_n)$ in $A$ with $x_n \neq c$ and
  $y_n \neq c$ and
  \[
    \lim x_n = \lim y_n = c \quad \text{but} \quad \lim f(x_n) \neq \lim f(y_n),
  \]
  then we can conclude that the functional limit $\lim_{x \to c} f(x)$ does not
  exist.
\end{corollary}

\subsection*{4.3 Continuous Functions}

\begin{definition}[Continuity]
  A function $f : A \to \R$ is \emph{continuous at a point} $c \in A$ if, for
  all $\epsilon > 0$, there exists a $\delta > 0$ such that whenever $\abs{x -
  c} < \delta$ (and $x \in A$) it follows that $\abs{f(x) - f(c)} < \epsilon$.

  If $f$ is continuous at every point in the domain $A$, then we say that $f$ is
  \emph{continuous on $A$}.
\end{definition}

\begin{theorem}[Characterizations of Continuity]
  Let $f : A \to \R$, and let $c \in A$. The function $f$ is continuous at $c$
  if and only if any one of the following three conditions is met:
  \begin{enumerate}[label={(\roman*)}]
    \item For all $\epsilon > 0$, there exists a $\delta > 0$ such that $\abs{x
      - c} < \delta$ (and $x \in A$) implies $\abs{f(x) - f(c)} < \epsilon$;
    \item For all $V_\epsilon(f(c))$, there exists a $V_\delta(c)$ with the
      property that $x \in V_\delta(c)$ (and $x \in A$) implies $f(x) \in
      V_\epsilon(f(c))$;
    \item For all $(x_n) \to c$ (with $x_n \in A$), it follows that $f(x_n) \to
      f(c)$.
    \item If $c$ is a limit point of $A$, then the above conditions are
      equivalent to $\lim_{x \to c} f(x) = f(c)$.
  \end{enumerate}
\end{theorem}

\begin{corollary}[Criterion for Discontinuity]
  Let $f : A \to \R$, and let $c \in A$ be a limit point of $A$. If there exists
  a sequence $(x_n) \subseteq A$ where $(x_n) \to c$ but such that $f(x_n)$ does
  not converge to $f(c)$, we may conclude that $f$ is not continuous at $c$.
\end{corollary}

\begin{theorem}[Algebraic Continuity Theorem]
  Assume $f : A \to \R$ and $g : A \to \R$ are continuous at point $c \in A$.
  Then,
  \begin{enumerate}[label={(\roman*)}]
    \item $k f(x)$ is continuous at $c$ for all $k \in \R$;
    \item $f(x) + g(x)$ is continuous at $c$;
    \item $f(x) g(x)$ is continuous at $c$; and
    \item $f(x) / g(x)$ is continuous at $c$, provided the quotient is defined.
  \end{enumerate}
\end{theorem}

\subsection*{4.4 Continuous Functions on Compact Sets}

\begin{theorem}[Preservation of Compact Sets]
  Let $f : A \to \R$ be continuous on $A$. If $K \subseteq A$ is compact, then
  $f(K)$ is compact as well.
\end{theorem}

\begin{theorem}[Extreme Value Theorem]
  If $f : K \to \R$ is continuous on a compact set $K \subseteq \R$, then $f$
  attains a maximum and minimum value. In other words, there exists $x_0, x_1
  \in K$ such that $f(x_0) \le f(x) \le f(x_1)$ for all $x \in K$.
\end{theorem}

\begin{definition}[Uniform Continuity]
  A function $f : A \to \R$ is \emph{uniformly continuous on $A$} if for every
  $\epsilon > 0$ there exists a $\delta > 0$ such that for all $x, y \in A$,
  $\abs{x - y} < \delta$ implies $f(x) - f(y) < \epsilon$.
\end{definition}

\begin{theorem}[Sequential Criterion for Absence of Uniform Continuity]
  A function $f : A \to \R$ fails to be uniformly continuous on $A$ if and only
  if there exists a particular $\epsilon_0 > 0$ and two sequences $(x_n)$ and
  $(y_n)$ in $A$ satisfying
  \[
    \abs{x_n - y_n} \to 0 \quad \text{but} \quad \abs{f(x_n) - f(y_n)} \ge
    \epsilon_0.
  \]
\end{theorem}

\begin{theorem}[Uniform Continuity on Compact Sets]
  A function that is continuous on a compact set $K$ is uniformly continuous on
  $K$.
\end{theorem}

\subsection*{4.5 The Intermediate Value Theorem}

\begin{theorem}[Intermediate Value Theorem]
  Let $f : [a, b] \to \R$ be continuous. If $L$ is a real number satisfying
  $f(a) < L < f(b)$ or $f(a) > L > f(b)$, then there exists a point $c \in (a,
  b)$ where $f(c) = L$.
\end{theorem}

\section*{5 The Derivative}

\subsection*{5.2 Derivatives and the Intermediate Value Property}

\begin{definition}[Differentiability]
  Let $g : A \to \R$ be a function defined on an interval $A$. Given $c \in A$,
  the \emph{derivative of $g$ at $c$} is defined by
  \[
    g'(c) = \lim_{x \to c} \frac{g(x) - g(c)}{x - c},
  \]
  provided this limit exists. In this case we say \emph{$g$ is differentiable at
  $c$}. If $g'$ exists for all points $c \in A$, we say that \emph{$g$ is
  differentiable on $A$}.
\end{definition}

\begin{theorem}
  If $g : A \to \R$ is differentiable at point $c \in A$, then $g$ is continuous
  at $c$ at well.
\end{theorem}

\begin{theorem}[Algebraic Differentiability Theorem]
  Let $f$ and $g$ be functions defined on an interval $A$, and assume both are
  differentiable at some point $c \in A$. Then,
  \begin{enumerate}[label={(\roman*)}]
    \item $(f + g)'(c) = f'(c) + g'(c)$,
    \item $(kf)'(c) = k f'(c)$, for all $k \in \R$,
    \item $(fg)'(c) = f'(c) g(c) + f(c) g'(c)$, and
    \item $(f/g)'(c) = \frac{g(c) f'(c) - f(c) g'(c)}{{[g(c)]}^2}$, provided
      that $g(c) \neq 0$.
  \end{enumerate}
\end{theorem}

\begin{theorem}[Chain Rule]
  Let $f : A \to \R$ and $g : B \to \R$ satisfy $f(A) \subseteq B$ so that the
  composition $g \circ f$ is defined. If $f$ is differentiable at $c \in A$ and
  if $g$ is differentiable at $f(c) \in B$, then $g \circ f$ is differentiable
  at $c$ with $(g \circ f)'(c) = g'(f(c)) \cdot f'(c)$.
\end{theorem}

\begin{theorem}[Interior Extremum Theorem]
  Let $f$ be differentiable on an open interval $(a, b)$. If $f$ attains a
  maximum value at some point $c \in (a, b)$ (i.e., $f(c) \ge f(x)$ for all $x
  \in (a, b)$), then $f'(c) = 0$. The same is true if $f(c)$ is a minimum value.
\end{theorem}

\subsection*{5.3 The Mean Value Theorems}

\begin{theorem}[Rolle's Theorem]
  Let $f : [a, b] \to \R$ be continuous on $[a, b]$ and differentiable on $(a,
  b)$. If $f(a) = f(b)$, then there exists a point $c \in (a, b)$ where $f'(c) =
  0$.
\end{theorem}

\begin{theorem}[Mean Value Theorem]
  If $f : [a, b] \to \R$ is continuous on $[a, b]$ and differentiable on $(a,
  b)$, then there exists a point $c \in (a, b)$ where
  \[
    f'(c) = \frac{f(b) - f(a)}{b - a}.
  \]
\end{theorem}

\begin{corollary}
  If $g : A \to \R$ is differentiable on an interval $A$ and satisfies $g'(x) =
  0$ for all $x \in A$, then $g(x) = k$ for some constant $k \in \R$.
\end{corollary}

\begin{corollary}
  If $f$ and $g$ are differentiable functions on an interval $A$ and satisfy
  $f'(x) = g'(x)$ for all $x \in A$, then $f(x) = g(x) + k$ for some constant $k
  \in \R$.
\end{corollary}

\section*{6 Sequences and Series of Functions}

\subsection*{6.2 Uniform Convergence of a Sequence of Functions}

\begin{definition}[Pointwise Convergence]
  For each $n \in \N$, let $f_n$ be a function defined on a set $A \subseteq
  \R$. The sequence $(f_n)$ of functions \emph{converges pointwise on $A$} to a
  function $f$ if, for all $x \in A$, the sequence of real numbers $f_n(x)$
  converges to $f(x)$.

  In this case, we write $f_n \to f$, $\lim f_n = f$, or $\lim_{n \to \infty}
  f_n(x) = f(x)$. This last expression is helpful if there is any confusion as
  to whether $x$ or $n$ is the limiting variable.
\end{definition}

\begin{definition}[Pointwise Convergence]
  Let $(f_n)$ be a sequence of functions defined on a set $A \subseteq \R$.
  Then, $(f_n)$ \emph{converges pointwise on $A$} to a limit $f$ defined on $A$
  if, for every $\epsilon > 0$ and $x \in A$, there exists an $N \in \N$
  (perhaps dependent on $x$) such that $\abs{f_n(x) - f(x)} < \epsilon$ whenever
  $n \ge N$.
\end{definition}

\begin{definition}[Uniform Convergence]
  Let $(f_n)$ be a sequence of functions defined on a set $A \subseteq \R$.
  Then, $(f_n)$ \emph{converges uniformly on $A$} to a limit function $f$
  defined on $A$ if, for every $\epsilon > 0$, there exists an $N \in \N$ such
  that $\abs{f_n(x) - f(x)} < \epsilon$ whenever $n \ge N$ and $x \in A$.
\end{definition}

\begin{theorem}[Cauchy Criterion for Uniform Convergence]
  A sequence of functions $(f_n)$ defined on a set $A \subseteq \R$ converges
  uniformly on $A$ if and only if for every $\epsilon > 0$ there exists an $N
  \in \N$ such that $\abs{f_n(x) - f_m(x)} < \epsilon$ whenever $m, n \ge N$ and
  $x \in A$.
\end{theorem}

\begin{theorem}[Continuous Limit Theorem]
  Let $(f_n)$ be a sequence of functions defined on $A \subseteq \R$ that
  converges uniformly on $A$ to a function $f$. If each $f_n$ is continuous at
  $c \in A$, then $f$ is continuous at $c$.
\end{theorem}

\subsection*{6.3 Uniform Convergence and Differentiation}

\begin{theorem}[Differentiable Limit Theorem]
  Let $f_n \to f$ pointwise on the closed interval $[a, b]$, and assume that
  each $f_n$ is differentiable. If $(f_n')$ converges uniformly on $[a, b]$ to a
  function $g$, then the function is differentiable and $f' = g$.
\end{theorem}

\subsection*{6.4 Series of Functions}

\begin{definition}
  For each $n \in \N$, let $f_n$ and $f$ be functions defined on a set $A
  \subseteq \R$. The infinite series
  \[
    \sum_{n=1}^\infty f_n(x) = f_1(x) + f_2(x) + f_3(x) + \cdots
  \]
  \emph{converges pointwise on $A$} to $f(x)$ if the sequence $s_k(x)$ of
  partial sums defined by
  \[
    s_k(x) = f_1(x) + f_2(x) + \cdots + f_k(x)
  \]
  converges pointwise to $f(x)$. The series \emph{converges uniformly on $A$} to
  $f$ if the sequence $s_k(x)$ converges uniformly on $A$ to $f(x)$.

  In either case, we write $f = \sum_{n=1}^\infty f_n$ or $f(x) =
  \sum_{n=1}^\infty f_n(x)$, always being explicit about the type of convergence
  involved.
\end{definition}

\begin{theorem}[Term-by-term Continuity Theorem]
  Let $f_n$ be continuous functions defined on a set $A \subseteq \R$, and
  assume $\sum_{n=1}^\infty f_n$ converges uniformly on $A$ to a function $f$.
  Then, $f$ is continuous on $A$.
\end{theorem}

\begin{theorem}[Term-by-term Differentiability Theorem]
  Let $f_n$ be differentiable functions defined on an interval $A$, and assume
  $\sum_{n=1}^\infty f_n'(x)$ converges uniformly to a limit $g(x)$ on $A$. If
  $\sum_{n=1}^\infty f_n(x)$ converges pointwise to $f(x)$, then $f(x)$ is
  differentiable and $f'(x) = g(x)$ on $A$.
\end{theorem}

\begin{theorem}[Cauchy Criterion for Uniform Convergence of Series]
  A series $\sum_{n=1}^\infty f_n$ converges uniformly on $A \subseteq \R$ if
  and only if for every $\epsilon > 0$ there exists an $N \in \N$ such that
  \[
    \abs{f_{m+1}(x) + f_{m+2}(x) + f_{m+3}(x) + \cdots + f_n(x)} < \epsilon
  \]
  whenever $n > m \ge N$ and $x \in A$.
\end{theorem}

\begin{corollary}[Weierstrass M-Test]
  For each $n \in \N$, let $f_n$ be a function defined on a set $A \subseteq
  \R$, and let $M_n > 0$ be a real number satisfying
  \[
    \abs{f_n(x)} \ge M_n
  \]
  for all $x \in A$. If $\sum_{n=1}^\infty M_n$ converges, then
  $\sum_{n=1}^\infty f_n$ converges uniformly on $A$.
\end{corollary}

\end{document}
